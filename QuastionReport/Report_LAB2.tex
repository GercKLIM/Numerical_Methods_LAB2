\documentclass[12pt, a4paper]{article}

\usepackage[utf8]{inputenc}
\usepackage[T2A]{fontenc}
\usepackage[russian]{babel}
\usepackage[dvips]{graphicx}

\usepackage[oglav,spisok,boldsect,eqwhole,figwhole,hyperref,hyperprint,remarks,greekit]{./style/fn2kursstyle}
\graphicspath{{./style/}{./figures/}}
\usepackage{float}
\usepackage{multirow}
\usepackage{supertabular}
\usepackage{multicol}
\usepackage{hhline}
\usepackage{listings}
\usepackage{color}
\usepackage{adjustbox}
\usepackage{amsmath}
\usepackage{verbatim}

\definecolor{dkgreen}{rgb}{0,0.6,0}
\definecolor{gray}{rgb}{0.5,0.5,0.5}
\definecolor{mauve}{rgb}{0.58,0,0.82}

\lstset{frame=tb,
	language=C++,
	aboveskip=1mm,
	belowskip=1mm,
	showstringspaces=false,
	columns=flexible,
	basicstyle={\small},
	numbers=left,
	numberstyle=\tiny\color{gray},
	keywordstyle=\color{red},
	commentstyle=\color{dkgreen},
	stringstyle=\color{mauve},
	breaklines=true,
	breakatwhitespace=true,
	tabsize=2
}
\title{Численное решение краевых задач для одномерного уравнения теплопроводности \\ Варианты 5, 16}


%\authorfirst{О.\,Д.~Климов}
%\authorsecond{О.\,Д.~Климов} TODO: прописать команды в style.sty

\supervisor{С.\,А.~Конев}
\group{ФН2-61Б}
\date{2024}

%\renewcommand{\vec}[1]{\text{\mathversion{bold}${#1}$}}%{\bi{#1}}
\newcommand\thh[1]{\text{\mathversion{bold}${#1}$}}
\renewcommand{\labelenumi}{\theenumi)}
\renewcommand{\labelenumi}{\theenumi)}

\newcommand{\opr}{\textbf{\underline{{Опр.}}}\quad}
\newcommand{\theorem}{\textbf{\underline{{Теор.}}}\quad}
\renewcommand{\phi}{\varphi}
\renewcommand{\k}[1]{\textbf{\textit{#1}}}

\newcounter{mycounter}
\newcommand{\quastion}[1]{%
	\stepcounter{mycounter}%
	\textbf{\themycounter}.  %
	\textbf{\textit{#1}}
	
}

\newcommand{\rusg}{\text{Г}}

\begin{document}
	\maketitle
	\section{Ответы на контрольные вопросы}
	
	\quastion{Дайте определения терминам: корректно поставленная задача, понятие аппроксимации дифференциальной задачи разностной схемой, порядок аппроксимации, однородная схема, консервативная схема, монотонная схема, устойчивая разностная схема (условно/абсолютно), сходимость.}
	
	Пусть мы рассматриваем задачу о нахождении решения уравнения $Au = f$ в области $G$ с дополнительными условиями $Ru = \mu$ на границе $\rusg = \partial G$ области $G$:
	\begin{equation}
		Au = f   \text{ в } \, G, \quad Ru = \mu \text{ на } \rusg.
		\label{1}
	\end{equation}
	где $A$, $R$ --- заданные операторы, $f$,$\mu$ --- заданные функции.
	
	\opr \textbf{Задача} называется \textbf{корректно поставленной}, если ее решение существует, единственно и непрерывно зависит от входных данных. Если же не выполнено хотя бы одно из этих условий, то задача называется некорректно поставленной.
	
	\opr Функция, определённая только в узлах сетки, называется сеточной.
	
	\opr Построение разностной схемы --- это замена уравнений и дополнительных условий исходной задачи алгебраическими уравнениями для сеточных функций. Производные в исходных уравнениях заменяют конечными разностями, интегралы --- квадратурными формулами, прочие члены --- алгебраическими соотношениями.
	
	Пусть для точной задачи \eqref{1} используется разностная схема
	\begin{equation}
		A_h y = \phi   \text{ в } \, G_h, \quad R_h y = \nu \text{ на } \Gamma_h.
		\label{2}
	\end{equation}
	Сеточные функции $\psi_h = (\phi - f_h) + ((A v)_h - A_h v_h)$, \quad $\chi_h = (\nu - \mu_h) + ((R v)_h - R_h v_h)$ --- \textbf{погрешности аппроксимации разностной} задачи в~$G_h$ и~на~$\Gamma_h$ соответственно ($v$ есть произвольная функция из области определения оператора $A$).
	
	\opr Говорят, что \textbf{разностная схема аппроксимирует исходную задачу}, если $ ||\psi_h||_\psi \to 0$, $ ||\chi_h||_\chi \to 0$ при $h \to 0$. Аппроксимация имеет \textbf{\textit{p}-й порядок} ($p > 0$), если $ ||\psi_h||_\psi = O(h^p)$, $ ||\chi_h||_\chi = O(h^p)$  при $h \to 0$.
	
	\opr Разностные схемы называются \textbf{консервативными}, если их решение удовлетворяет дискретному аналогу закона сохранения (баланса), присущему данной задаче. В противном случае схему называют \textbf{неконсервативной}, или дисбалансной.
	
	\opr Разностные схемы, в которых расчет ведется по одним формулам и на одном шаблоне во всех узлах сетки без какого-то либо специального выделения имеющихся особенностей, называются \textbf{однородными}. 
	
	\opr Схемы, решение которых удовлетворяет принципу максимума или сохраняет пространственную монотонность(в одномерном случае) при условии, что соответствующие свойства справедливы для исходных задач, называются \textbf{монотонными}. 
	
	Пусть $y_1$, $y_2$ --- решения двух задач с одинаковым оператором, соответствующие правым частям $\phi_1$, $\phi_2$ и граничным условиям $\nu_1$, $\nu_2$.
	
	\opr Говорят, что \textbf{разностная схема устойчивая}, если решение уравнений схемы непрерывно зависит от входных данных и эта зависимость равномерна по $h$, т.е.
	\begin{equation*}
		\forall \varepsilon > 0 \phantom{x}\exists \delta(\varepsilon) > 0: \quad \parallel \phi_1 - \phi_2 \parallel_\phi < \delta,   \parallel \nu_1 - \nu_2 \parallel_\nu <  \delta \Rightarrow \parallel y_1 - y_2 \parallel_Y < \varepsilon
	\end{equation*}	
	
	\opr В случае нескольких независимых переменных \textbf{устойчивость} называют \textbf{безусловной}(или \textbf{абсолютной}), если устойчивость имеет место для любого соотношения шагов, и \textbf{условной} в противном случае.
	
	Будем различать следующие устойчивости. Устойчивость по правой части: непрерывная зависимость решения разностной задачи от $\phi$. Устойчивость по граничным условиям: непрерывная зависимость решения разностной задачи от $\nu$ на границе пространственной области. Устойчивость по начальным данным: непрерывная зависимость решения разностной задачи от $\nu$ на гиперплоскости $t=0$.
	
	\opr Разностное решение $y$ сходится к решению $u$ точной задачи, если \\
	\mbox{$\| y - p_h u \|_Y \to 0$} при $h \to 0$. Говорят, что имеет место \textbf{сходимость} с \textit{p}-м ($p > 0$) порядком, если $\| y - p_h u \|_Y = O(h^p)$ при $h \to 0$.
	
	\textsl{Замечание.} $f(x) = O(g(x)) \Leftrightarrow \forall x \in U(x): |f(x)| \le C|g(x)|$.
	
	
	\clearpage % для удобства редактирования <-> потом убрать
	\quastion{Какие из рассмотренных схем являются абсолютно устойчивыми? Какая из рассмотренных схем позволяет вести расчеты с более крупным шагом по времени?}
	
	Схема Дюфорта---Франсла
	
	
	\clearpage % для удобства редактирования <-> потом убрать
	\quastion{Будет ли смешанная схема иметь второй порядок аппроксимации при $a_i = \frac{2 K(x_i) K(x_{i-1})}{K(x_i) + K(x_{i-1})}$?}
	
	
	% ХЗ верно ли: Вариант исследования схемы, где мы приводим ее к формуле симпсона, но мне она кажется неверной из-за того, что подрузамеватся, что $h = x_i - x_{i+1}$, в то время как в методичке полагается $h = x_{i- 0.5} - x_{i+0.5} $
	\begin{comment}
		Имем однородную консервативную схему 
		\begin{equation*}
			c \rho \frac{y^{j+1}_i - y^{j}_i}{\tau} = \frac{1}{h} (\sigma(a_{i+1} y^{j+1}_{i+\frac{1}{2}} - a_{i} y^{j+1}_{i-\frac{1}{2}}) + (1-\sigma)(a_{i+1} y^{j}_{i+\frac{1}{2}} - a_{i} y^{j}_{i-\frac{1}{2}})),
		\end{equation*}
		где $a_i = \displaystyle{(\frac{1}{h}\int_{x_{i-1}}^{x_i} \frac{dx}{K(x)})^{-1}}$.
		
		%%% $ c \rho \frac{y^{j+1}_i - y^{j}_i}{\tau} = \frac{1}{h} (\sigma(\omega^{j+1}_{i+\frac{1}{2}} - \omega^{j+1}_{i-\frac{1}{2}}) + (1-\sigma)(\omega^{j}_{i+\frac{1}{2}} - \omega^{j}_{i-\frac{1}{2}}))$
		 
		Пусть $a_i = \frac{2 K(x_i) K(x_{i-1})}{K(x_i) + K(x_{i-1})}$. Тогда из равенства $\frac{2 K(x_i) K(x_{i-1})}{K(x_i) + K(x_{i-1})} = \displaystyle{(\frac{1}{h}\int_{x_{i-1}}^{x_i} \frac{dx}{K(x)})^{-1}}$ получим, что
		\begin{equation*}
			\int_{x_{i-1}}^{x_i} \frac{dx}{K(x)} = h \frac{2 K(x_i) K(x_{i-1})}{K(x_i) + K(x_{i-1})} = \frac{h}{2} (\frac{1}{K(x_{i-1})} + \frac{1}{K(x_{i})}).
		\end{equation*}
		Можно увидеть, что выражение является формулой трапеций, которая, как известно, имеет второй порядок точности. Следовательно рассмотренный случай также имеет 2 порядок.
	\end{comment}
	
	Имеем однородную консервативную схему 
	\begin{equation*}
		c \rho \frac{y^{j+1}_i - y^{j}_i}{\tau} = \frac{1}{h} (\sigma(a_{i+1} y^{j+1}_{i+\frac{1}{2}} - a_{i} y^{j+1}_{i-\frac{1}{2}}) + (1-\sigma)(a_{i+1} y^{j}_{i+\frac{1}{2}} - a_{i} y^{j}_{i-\frac{1}{2}})),
	\end{equation*}
	Преобразовав данное в вопросе выражение, имеем
	\begin{equation*}
		\frac{1}{a_i} = \frac{1}{2} \left( \frac{1}{K_i} + \frac{1}{K_{i-1}} \right)
	\end{equation*}
	
	
	Разложим в ряд Тейлора в окрестности точки $x_{i-\frac{1}{2}} = x_i - \frac{h}{2}$:
	\begin{equation*}
		\frac{1}{K_i} = \frac{1}{K_{i-\frac{1}{2}}} - \frac{2 K'_{i-\frac{1}{2}}}{(K_{i-\frac{1}{2}})^2} \frac{1}{2!} \frac{h}{2} + O(h^2), \quad
		\frac{1}{K_{i-\frac{1}{2}}} = \frac{1}{K_{i-\frac{1}{2}}} + \frac{2 K'_{i-\frac{1}{2}}}{(K_{i-\frac{1}{2}})^2} \frac{1}{2!} \frac{h}{2} + O(h^2)
	\end{equation*}
	Подставим и получим
	\begin{equation*}
	\frac{1}{a_i} = \frac{1}{2} \left( \frac{1}{K_i} + \frac{1}{K_{i-1} } + O(h^2)\right) = \frac{1}{K_{i-\frac{1}{2}}} + O(h^2)  \quad \Rightarrow \quad a_i = K_{i-\frac{1}{2}} + O(h^2)
	\end{equation*}
	Следовательно при $K_{i-\frac{1}{2}}$ получаем порядок аппроксимации $O(h^2)$.
	
	
	
	
	\clearpage % для удобства редактирования <-> потом убрать
	\quastion{Какие методы (способы) построения разностной аппроксимации граничных условий (2.5), (2.6) с порядком точности $O(\tau + h^2)$, $O(\tau^2 + h^2)$, $O(\tau^ + h)$ вы знаете?}
	
	\clearpage % для удобства редактирования <-> потом убрать
	\quastion{При каких $h$, $\tau$ и $\sigma$ смешанная схема монотонна? Проиллюстрируйте результатами расчетов свойства монотонных и немонотонных разностных схем.}
	
	
	
	
	
	
	
	
	
	
	
	
	
	
	
	
	
	
	
	
	
	
	
	\clearpage % для удобства редактирования <-> потом убрать
	\quastion{Какие ограничения на $h$, $\tau$ и $\sigma$ накладывают условия устойчивости прогонки?}
	
	\clearpage % для удобства редактирования <-> потом убрать
	\quastion{В случае $K = K(u)$ чему равно количество внутренних итераций, если итерационный процесс вести до сходимости, а не обрывать после нескольких первых итераций?}
	
	\clearpage % для удобства редактирования <-> потом убрать
	\quastion{Для случая K = K(u) предложите способы организации внутреннего итерационного процесса или алгоритмы, заменяющие его.}
	
	
	
	
	
	\clearpage
	\begin{thebibliography}{1}
		\bibitem{1} Галанин М.П., Савенков Е.Б. Методы численного анализа математических моделей. М.: Изд-во МГТУ им. Н.Э. Баумана. 2018. 592 с.
		
	\end{thebibliography}
	
\end{document}