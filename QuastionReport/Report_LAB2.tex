\documentclass[12pt, a4paper]{article}

\usepackage[utf8]{inputenc}
\usepackage[T2A]{fontenc}
\usepackage[russian]{babel}
\usepackage[dvips]{graphicx}

\usepackage[oglav,spisok,boldsect,eqwhole,figwhole,hyperref,hyperprint,remarks,greekit]{./style/fn2kursstyle}
\graphicspath{{./style/}{./figures/}}
\usepackage{float}
\usepackage{multirow}
\usepackage{supertabular}
\usepackage{multicol}
\usepackage{hhline}
\usepackage{listings}
\usepackage{color}
\usepackage{adjustbox}
\usepackage{amsmath}

\definecolor{dkgreen}{rgb}{0,0.6,0}
\definecolor{gray}{rgb}{0.5,0.5,0.5}
\definecolor{mauve}{rgb}{0.58,0,0.82}

\lstset{frame=tb,
	language=C++,
	aboveskip=1mm,
	belowskip=1mm,
	showstringspaces=false,
	columns=flexible,
	basicstyle={\small},
	numbers=left,
	numberstyle=\tiny\color{gray},
	keywordstyle=\color{red},
	commentstyle=\color{dkgreen},
	stringstyle=\color{mauve},
	breaklines=true,
	breakatwhitespace=true,
	tabsize=2
}
\title{Численное решение краевых задач для одномерного уравнения теплопроводности \\ Варианты 5, 16}


%\authorfirst{О.\,Д.~Климов}
%\authorsecond{О.\,Д.~Климов} TODO: прописать команды в style.sty

\supervisor{С.\,А.~Конев}
\group{ФН2-61Б}
\date{2024}

%\renewcommand{\vec}[1]{\text{\mathversion{bold}${#1}$}}%{\bi{#1}}
\newcommand\thh[1]{\text{\mathversion{bold}${#1}$}}
\renewcommand{\labelenumi}{\theenumi)}
\renewcommand{\labelenumi}{\theenumi)}

\newcommand{\opr}{\textbf{\underline{{Опр.}}}\quad}
\newcommand{\theorem}{\textbf{\underline{{Теор.}}}\quad}
\renewcommand{\phi}{\varphi}
\renewcommand{\k}[1]{\textbf{\textit{#1}}}

\newcounter{mycounter}
\newcommand{\quastion}[1]{%
	\stepcounter{mycounter}%
	\textbf{\themycounter}.  %
	\textbf{\textit{#1}}
	
}


\begin{document}
	\maketitle
	\section{Ответы на контрольные вопросы}
	
	\quastion{Дайте определения терминам: корректно поставленная задача, понятие аппроксимации дифференциальной задачи разностной схемой, порядок аппроксимации, однородная схема, консервативная схема, монотонная схема, устойчивая разностная схема (условно/абсолютно), сходимость.}
	
	\quastion{Какие из рассмотренных схем являются абсолютно устойчивыми? Какая из рассмотренных схем позволяет вести расчеты с более крупным шагом по времени?}
	
	\quastion{Будет ли смешанная схема иметь второй порядок аппроксимации при $a_i = \frac{2 K(x_i) K(x_{i-1})}{K(x_i) + K(x_{i-1})}$?}
	\quastion{Какие методы (способы) построения разностной аппроксимации граничных условий (2.5), (2.6) с порядком точности $O(\tau + h^2)$, $O(\tau^2 + h^2)$, $O(\tau^ + h)$ вы знаете?}
	\quastion{При каких $h$, $\tau$ и $\sigma$ смешанная схема монотонна? Проиллюстрируйте результатами расчетов свойства монотонных и немонотонных разностных схем.}
	\quastion{Какие ограничения на $h$, $\tau$ и $\sigma$ накладывают условия устойчивости прогонки?}
	\quastion{В случае $K = K(u)$ чему равно количество внутренних итераций, если итерационный процесс вести до сходимости, а не обрывать после нескольких первых итераций?}
	\quastion{Для случая K = K(u) предложите способы организации внутреннего итерационного процесса или алгоритмы, заменяющие его.}
	
	
	
	
	
	\clearpage
	\begin{thebibliography}{1}
		\bibitem{1} Галанин М.П., Савенков Е.Б. Методы численного анализа математических моделей. М.: Изд-во МГТУ им. Н.Э. Баумана. 2018. 592 с.
		
	\end{thebibliography}
	
\end{document}